Package for controlling the real Mia Hand within R\+OS Control. The cpp class Mia\+H\+W\+Interface inherits from public hardware\+\_\+interface\+::\+Robot\+HW and allows to control the Mia Hand coupled fingers using the R\+OS controllers. In this package a specific transmission class has been declared for each of the three actuators of the Mia hand due to their non linear and peculiar behaviours.

To launch the real mia hand it is necessary to launch the Mia\+\_\+hw\+\_\+node that implements the actual control loop. To do that one of the following pre-\/tested commands can be used\+:


\begin{DoxyItemize}
\item to launch the control loop (frequency 100Hz) with one controller (position controller as default) for each DoF of the Mia Hand, attaching the Mia hand at C\+OM 0 (default)\+: \begin{DoxyVerb}roslaunch mia_hand_bringup mia_hand_hw.launch
\end{DoxyVerb}


Other inputs arguments of the above launch file are\+: \begin{DoxyVerb}Mia_COM_:               default='0'            doc="Int number of the Mia hand COM port"
Mia_fs_:                default='100'          doc="Hz, frequency of the control loop of the Mia Hw interface"
controller_thumb_fle    default="position"     doc="type of feed forward controller for thumb flexion. Values: position, velocity"
controller_index_fle    default="position"     doc="type of feed forward controller for indexe flexion. Values: position, velocity"
controller_mrl_fle      default="position"     doc="type of feed forward controller for mrl flexion. Values: position, velocity"
robotNamespace          default="mia_hand_hw"  doc="Namespace of the robot"
\end{DoxyVerb}

\item to launch the control loop (frequency 100Hz) with a traj controller (velocity traj controller as default) that control all the DoF of the Mia Hand, attaching the Mia hand at C\+OM 0 (default)\+: \begin{DoxyVerb}roslaunch mia_hand_bringup mia_hand_hw_traj.launch
\end{DoxyVerb}


Other inputs arguments of the above launch file are\+: \begin{DoxyVerb}Mia_COM_:               default='0'            doc="Int number of the Mia hand COM port"
Mia_fs_:                default='100'          doc="Hz, frequency of the control loop of the Mia Hw interface"
tc_type:                default="vel"          doc="type of trajectory controller to launch. Values: vel, pos"
robotNamespace          default="mia_hand_hw"  doc="Namespace of the robot"
\end{DoxyVerb}

\item to launch the control loop (frequency 100Hz) with a traj controller (velocity traj controller as default) and the R\+V\+IZ Moveit G\+UI, attaching the Mia hand at C\+OM 0 (default)\+: \begin{DoxyVerb}roslaunch mia_hand_bringup mia_hand_hw_moveit.launch
\end{DoxyVerb}


Other inputs arguments of the above launch file are\+: \begin{DoxyVerb}Mia_COM_:               default='0'            doc="Int number of the Mia hand COM port"
Mia_fs_:                default='100'          doc="Hz, frequency of the control loop of the Mia Hw interface"
tc_type:                default="vel"          doc="type of trajectory controller to launch. Values: vel, pos"
robotNamespace          default="mia_hand_hw"  doc="Namespace of the robot"
\end{DoxyVerb}


NB\+: to run this command the mia\+\_\+hand\+\_\+moveit\+\_\+config package is needed. When the R\+V\+Iz turns on it is possible to use the moveit plug-\/in to control the \mbox{\hyperlink{namespacemia__hand}{mia\+\_\+hand}} subgroup chains (thumb\+\_\+flexion, index\+\_\+flexion and mrl\+\_\+flexion). For the index two different links will be displayed (if the option to visualize the moveit goal position is selected)\+: one index link displays the signed target position (for exaple index in position -\/1.\+2 rad), the other index link displays the actual configuration of the index ( so after having reached the target position of -\/1.\+2 rad this index link will be at pos 1.\+2 rad flexed and the thumb will be adducted). 
\end{DoxyItemize}